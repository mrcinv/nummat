\documentclass[slovene]{article}
\usepackage[T1]{fontenc}
\usepackage[utf8x]{inputenc}
\usepackage{amsmath}
\DeclareMathOperator*\sn{sn}
\usepackage{listings}
\lstset{basicstyle={\footnotesize},
frame=lines,
language=Octave,
literate={{č}{{\v{c}}}1 {š}{{\v{s}}}1 {ž}{{\v{z}}}1}}
\usepackage{textcomp}
\usepackage{babel}
\begin{document}

\title{Izpit iz Numerične matematike}


\date{26. 9. 2014}
\maketitle
\begin{enumerate}
\item Funkcijo $\sin(x)$ lahko na intervalu $[-1,1]$ zelo natančno interpoliramo s polinomi, če za interpolacijske točke uporabimo Čebiševe točke
$$
x_i=\cos(\frac{k\pi}{n}); \quad k=0,\ldots,n.
$$ 
Za polinoma stopnje $2$ in $20$ analitično oceni in numerično določi napako interpolacije v Čebiševih točkah. 

Bonus. Koliko računskih operacij bi potrebovali za izračun funkcije $sin$ 
v dvojni natančnosti, če bi jo računali z interpolacijskim polinomom v Čebiševih točkah? 

\item Eliptična funkcija $y=\sn(x,m)$ je definirana z relacijo
\begin{equation}
\sin(\phi)=\sn\left(\int_0^\phi\frac{dt}{\sqrt{1-m\sin^2t}},m\right).\label{eq:sn}
\end{equation}

Izračunaj približek za $\mathrm{sn}(0.5,2)$, ki bo imel vsaj 5 pravilnih števk. \emph{Namig: Z eno od 
numeričnih metod za nelinearne enačbe poišči $\phi$, ki reši enačbo
\begin{equation}
  \label{eq:enacba}
   x = 0.5 = \int_0^\phi\frac{dt}{\sqrt{1-2\sin^2t}}.
\end{equation}}
\end{enumerate}
\end{document}
