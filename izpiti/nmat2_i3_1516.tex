% Created 2016-08-23 tor 11:33
\documentclass[11pt]{article}
\usepackage[utf8]{inputenc}
\usepackage[T1]{fontenc}
\usepackage{fixltx2e}
\usepackage{graphicx}
\usepackage{grffile}
\usepackage{longtable}
\usepackage{wrapfig}
\usepackage{rotating}
\usepackage[normalem]{ulem}
\usepackage{amsmath}
\usepackage{textcomp}
\usepackage{amssymb}
\usepackage{capt-of}
\usepackage{hyperref}
\usepackage{minted}
\usepackage[slovene]{babel}
\usepackage{color}
\author{Martin Vuk}
\date{\today}
\title{Izpit iz numerične matematike\\\medskip
\large Ljubljana, 16. 6. 2016}
\hypersetup{
 pdfauthor={Martin Vuk},
 pdftitle={Izpit iz numerične matematike},
 pdfkeywords={},
 pdfsubject={},
 pdfcreator={Emacs 24.5.1 (Org mode 8.3.4)}, 
 pdflang={Slovene}}
\begin{document}

\maketitle
\tableofcontents


\section{Prva naloga}
\label{sec:orgheadline3}
V ravnini je krivulja $C$ podana z enačbo
$$(x^2+y^2)^2-2(x^2-y^2)=1.$$
\begin{enumerate}
\item Poišči vse točke na krivulji s koordinato $x=0.5$.
\item Poišči vsa presečišča krivulje $C$ in hiperbole
$$x^2-xy-y^2=1.$$ 
\end{enumerate}

\subsection{Rešitev}
\label{sec:orgheadline2}
Da si lažje predstavljamo, najprej krivuljo narišemo z ukazom `contour`
\begin{minted}[]{octave}
x = linspace(-2,2);
y  = linspace(-1,1);
[X,Y] = meshgrid(x,y);
figure( 1, "visible", "off" );
contour(X,Y, (X.^2+Y.^2).^2-2*(X.^2-Y.^2),[1,1],'b')
grid
print -dpng "krivuljaC.png"
\end{minted}

\includegraphics[width=.9\linewidth]{krivuljaC.png} 

Vidimo, da sta točki s koordinato \(x=0.5\) dve. Koordinata \(y\) je rešitev enačbe,
ki jo dobimo, če \(x=0.5\) vstavimo v enačbo krivulje 
\begin{eqnarray*}
(x^2+y^2)^2-2(x^2-y^2)&=&1|_{x=0.5}\\
(0.25+y^2)^2-2(0.25-y^2)&=&1\\
\end{eqnarray*}
Obe rešitvi lahko poiščemo z eno od metod
za reševanje nelinearnih enačb. Tu bomo uporabili \textbf{tangentno metodo}:
\begin{minted}[]{octave}
format long
f = @(y) (0.25 + y^2)^2-2*(0.25-y^2)-1;
df = @(y) 2*(0.25 + y^2)*2*y + 4*y;
y1 = -1;
for i=1:5
  y1 = y1 - f(y1)/df(y1)
end
\end{minted}

\begin{verbatim}

> > y1 = -0.770833333333333
y1 = -0.700309552168078
y1 = -0.694338996684583
y1 = -0.694298790208768
y1 = -0.694298788396521
\end{verbatim}
\begin{minted}[]{octave}
y2 = 1;
for i=1:5
  y2 = y2 - f(y2)/df(y2)
end
\end{minted}

\begin{verbatim}

> > y2 =  0.770833333333333
y2 =  0.700309552168078
y2 =  0.694338996684583
y2 =  0.694298790208768
y2 =  0.694298788396521
\end{verbatim}
\begin{minted}[]{octave}
hold on
plot([0.5,0.5],[y1,y2],'ro')
print -dpng "krivuljaC_tocke.png"
\end{minted}
\begin{figure}[htb]
\centering
\includegraphics[width=.9\linewidth]{krivuljaC_tocke.png}
\caption{Točke na krivulji c z \(x=0.5\)}
\end{figure}

\subsubsection{Presečiše krivulje C in hiperbole}
\label{sec:orgheadline1}
Narišimo še hiperbolo
\begin{minted}[]{octave}
x = linspace(-2,2);
y  = linspace(-1,1);
[X,Y] = meshgrid(x,y);
figure( 2, "visible", "off" );
contour(X,Y, (X.^2+Y.^2).^2-2*(X.^2-Y.^2),[1,1],'b')
grid
hold on
contour(X,Y, X.^2-X.*Y-Y.^2,[1,1],'r')
print -dpng "krivuljaC_in_hiperbola.png"
\end{minted}
\begin{figure}[htb]
\centering
\includegraphics[width=.9\linewidth]{krivuljaC_in_hiperbola.png}
\caption{Krivulja \(C\) in hiperbola}
\end{figure}
Krivulji imata 4 presečišča, ki jih lahko poiščemo kot rešitve sistema enačb
\begin{eqnarray*}
x^2-xy-y^2&=&1\\ 
(x^2+y^2)^2-2(x^2-y^2)&=&1.
\end{eqnarray*}
Zopet lahko uporabimo tangentno metodo.
\begin{minted}[]{octave}
F = @(x,y)  [x^2-x*y-y^2-1; (x^2+y^2)^2-2*(x^2-y^2)-1];
JF = @(x,y) [2*x-y, -2*y-y; 2*(x^2+y^2)*2*x-4*x, 2*(x^2+y^2)*2*y+4*y];
xy = [1;-1];
XY = xy';
for i=1:10
  xy = xy - JF(xy(1),xy(2))\F(xy(1),xy(2));
  XY = [XY;xy'];
end
XY
\end{minted}

\begin{verbatim}
> > > >> XY =

   1.000000000000000  -1.000000000000000
   0.812500000000000  -0.812500000000000
   0.848408340919436  -0.708985263996359
   0.908945644434683  -0.692194309712894
   0.916124464438256  -0.689418440683320
   0.917634241836431  -0.689121815861298
   0.917796981234154  -0.689091012155289
   0.917814022017620  -0.689087891663280
   0.917815749906834  -0.689087576452288
   0.917815924464424  -0.689087544621826
   0.917815942091686  -0.689087541407643
\end{verbatim}
Ostala presečišča poiščemo tako, da uporabimo druge začetne približke.
\begin{minted}[]{octave}
# 2. tocka
xy2 = [-1;-1];
XY = xy2';
for i=1:10
   xy2 = xy2 - JF(xy2(1),xy2(2))\F(xy2(1),xy2(2));
   XY = [XY; xy2'];
end
XY((end-3):end,:)
# 3. tocka
xy3 = [-1;1];
for i=1:10
   xy3 = xy3 - JF(xy3(1),xy3(2))\F(xy3(1),xy3(2));
   XY = [XY; xy3'];
end
XY((end-3):end,:)
# 4. tocka
xy4 = [1;1];
for i=1:10
   xy4 = xy4 - JF(xy4(1),xy4(2))\F(xy4(1),xy4(2));
   XY = [XY; xy4'];
end
XY((end-3):end,:)
\end{minted}

\begin{verbatim}
> > > >> ans =

  -1.359721054982997  -0.465128678114521
  -1.359707053302619  -0.465142895739003
  -1.359710517071326  -0.465139376740527
  -1.359709659723895  -0.465140247647315
> > > >> ans =

  -0.917814022017620   0.689087891663280
  -0.917815749906834   0.689087576452288
  -0.917815924464424   0.689087544621826
  -0.917815942091686   0.689087541407643
> > > >> ans =

   1.359721054982997   0.465128678114521
   1.359707053302619   0.465142895739003
   1.359710517071326   0.465139376740527
   1.359709659723895   0.465140247647315
\end{verbatim}
Vse skupaj predstavimo še grafično:
\begin{minted}[]{octave}
plot([xy(1),xy2(1),xy3(1),xy4(1)],[xy(2),xy2(2),xy3(2),xy4(2)],'ro')
print -dpng "krivuljaC_in_hiperbola_presek.png"
\end{minted}
\begin{figure}[htb]
\centering
\includegraphics[width=.9\linewidth]{krivuljaC_in_hiperbola_presek.png}
\caption{Presečišča krivulje \(C\) in hiperbole}
\end{figure}

\section{Druga naloga}
\label{sec:orgheadline8}

Funkcija $y(t)$ je podana kot rešitev začetnega problema
\[y''(t)+2y'(t)+y(t)=\sin(t),\quad y(0)=2,y'(0)=0.\] 
\begin{enumerate}
\item Z Eulerjevo metodo s korakom $h=0.5$ poiščite približek za $y(1)$.
\item Z metodo po vaši izbiri poiščite približek za $y(3)$ na 5 decimalk natančno.
\item Izračunajte integral
\[\int_0^3y(t)dt\]
na $5$ decimlak natančno.
\end{enumerate}
\subsection{Rešitev}
\label{sec:orgheadline7}
Najprej enačbo 2. reda prepišemo v sistem enačb 1. reda, tako da vpeljemo novo
spremenljivko \(z=y'(t)\). Dobimo sistem enačb
\begin{eqnarray*}
 y'(t) &=& z(t)\\
 z'(t) &=& \sin(t)-2z(t)-y(t) 
\end{eqnarray*}
z začetnim pogojem \(y(0)=2\) in \(z(0)=0\).
\subsubsection{Eulerjeva metoda}
\label{sec:orgheadline4}
Narediti moramo dva koraka Eulerjeve metode:
\begin{eqnarray*}
 y_1 &=& y_0+h*f(t_0,y_0)\\
y_2 &=& y +h*f(t_0+h,y_1)
\end{eqnarray*}
\begin{minted}[]{octave}
f = @(t,y,z) [z; sin(t)-2*z-y];
y0 = [2;0];
t0 = 0;
h = 0.5;
y1 = y0 + h*f(t0,y0(1),y0(2))
y2 = y1 + h*f(t0+h,y1(1),y1(2))
\end{minted}

\begin{verbatim}

y1 =

   2
  -1
y2 =

   1.500000000000000
  -0.760287230697898
\end{verbatim}
Približek za \(y(1)\) se skriva v prvi komponenti spremenljivke `y2` in je enak \(1.5\).
\subsubsection{Izračun \(y(3)\)}
\label{sec:orgheadline5}
Uporabili bomo metodo Runge-Kutta 4. reda.
\begin{minted}[]{octave}
fun = @(t,y) f(t,y(1),y(2));
function [y,t] = rk4(f,t0,y0,tk,n)
h = (tk-t0)/n;
y = y0;
t = t0;
for i=1:n
  k1 = h*f(t0,y0); 
  k2 = h*f(t0+h/2,y0+k1/2);
  k3 = h*f(t0+h/2,y0+k2/2);
  k4 = h*f(t0+h, y0+k3);
  y0 = y0 + (k1+2*k2+2*k3+k4)/6;
  t0 = t0+h;
  y = [y y0];
  t = [t t0];
end
endfunction
\end{minted}
Rešitev poiščemo za dve različni velikosti koraka in ocenimo napako z
Richardsonovo ekstrapolacijo.
\begin{minted}[]{octave}
tk = 3;
n = 20;
[y1,t] = rk4(fun,t0,y0,tk,n);
n = 40;
[y2,t] = rk4(fun,t0,y0,tk,n);
# ocena za napako
napaka = (y1(1,end)-y2(1,end))/(2^4-1)
\end{minted}

\begin{verbatim}

napaka =    2.17186881819783e-07
\end{verbatim}
Narišimo še graf rešitve.
\begin{minted}[]{octave}
figure( 1, "visible", "off" );
plot(t,y2(1,:),'b')
grid
print -dpng "resitevDE.png"
\end{minted}
\begin{figure}[htb]
\centering
\includegraphics[width=.9\linewidth]{resitevDE.png}
\caption{Rešitev \(y(t)\)}
\end{figure}
\subsubsection{Integral}
\label{sec:orgheadline6}
Integral lahko izračunamo iz že izračunanih približkov za rešitev, tako da
uporabimo npr. Simpsonovo sestavljeno pravilo.
\begin{minted}[]{octave}
h=(tk-t0)/n;
utezi = [1 repmat([4 2],1,n/2-1) 4 1];
I = h/3*sum(utezi.*y2(1,:))
\end{minted}

\begin{verbatim}

I =  4.30709978113080
\end{verbatim}

Druga možnost je, da integral enostavno dodamo v diferencialno enačbo. Naj bo
\[w(t) = \int_{t_0}^t y(\tau)d\tau\]
Funkcija \(w(t)\) zadošča diferencialni enačbi 
\[w'(t)=y(t)\] in začetnemu pogoju \(w(t_0)=0\). Sistem dveh enačb za \(y\) in \(z\)
razširimo s \(w\)
\begin{eqnarray*}
  y'(t) &=& z(t)\\
 z'(t) &=& \sin(t)-2z(t)-y(t)\\
 w'(t) &=& y(t)
\end{eqnarray*}
in uporabimo metodo RK.
\begin{minted}[]{octave}
funi = @(t,y) [f(t,y(1),y(2));y(1)];
y0 = [2;0;0];
n = 40;
[yi,t] = rk4(funi,t0,y0,tk,n);
I = yi(3,end)
\end{minted}

\begin{verbatim}

I =  4.30710150580884
\end{verbatim}
\end{document}