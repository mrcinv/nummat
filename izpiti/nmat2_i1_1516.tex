\documentclass[slovene,12pt]{article}
\usepackage[T1]{fontenc}
\usepackage[utf8x]{inputenc}
\usepackage{mathpazo}
\usepackage{amsmath}
\DeclareMathOperator*\erf{erf}
\usepackage{listings}
\lstset{basicstyle={\footnotesize},
frame=lines,
language=Octave,
literate={{č}{{\v{c}}}1 {š}{{\v{s}}}1 {ž}{{\v{z}}}1}}
\usepackage{textcomp}
\usepackage{babel}
\addtolength{\textheight}{1truecm}
\addtolength{\voffset}{-1truecm}
\begin{document}
\begin{center}
\section*{Izpit iz Numerične matematike}

Ljubljana, 16. 6. 2016
\end{center}
\pagestyle{empty}

\begin{enumerate}
\item Funkcijo napake  
\[\erf(x)=\frac{2}{\sqrt{\pi}}\int_0^xe^{-t^2}dt\] 
računamo s sestavljenim Simpsonovim pravilom.
 \begin{enumerate}
 \item Izračunajte $\erf(0.5)$ s korakom $h=0.25$ in določite napako.
 \item Ocenite, kako velik je lahko še korak  da bo napaka
   manjša od $5\cdot 10^{-11}$ za vse $x\in[0,1]$ \footnote[1]{napaka
     sestavljenega Simpsonovega pravila za integral $f$
   na intervalu $[a,b]$ s korakom $h$ je
\[\frac{h^4}{180}(b-a)f^{(4)}(\xi);\quad \xi\in[a,b]\]}.
 \item Izračunajte $\erf(0.5)$ in $\erf^{-1}(0.5)$ na 10 decimalk natančno.
 \end{enumerate}

\item Dana je linearna diferencialna enačba
  \begin{eqnarray}
    \label{eq:nde}
    y''(t) + t y'(t) + y(t) = t.
  \end{eqnarray}
  Označimo z $y_1$ rešitev enačbe (\ref{eq:nde}), ki zadošča začetnim pogojem
  $y(0)=1$ in $y'(0)=0$ in z $y_2$ rešitev, ki zadošča pogojem $y_2(0)=0$ in $y_2'(0)=1$. 
\begin{enumerate}
   \item Izračunajte $y_1(1)$ in $y_2(1)$ s sredinsko metodo
     \[y_{n+1} = y_n +  hf\left(t_n+\frac{h}{2},y_{n}+\frac{h}{2}f(t_n,y_{n})\right)\]
     s fiksnim korakom $h$ na 5 decimalk natančno. Koliko korakov potrebujete?
     Koliko je red sredinske metode? Pibližno skicirajte grafa $y_1$ in $y_2$.
   \item Utemeljite, zakaj je 
\[y = \frac{t}{2} + C(y_1-\frac{t}{2}) + D(y_2-\frac{t}{2})\] 
splošna rešitev diferencialne enačbe
     (\ref{eq:nde}) in določite konstanti $C$ in $D$, da bo $y$ rešitev robnega
     problema za enačbo (\ref{eq:nde}) s homogenimi robnimi pogoji
     $y(0)=y(1)=0$. Približno skicirajte graf rešitve robnega problema. 
  \end{enumerate}
\end{enumerate}
\end{document}

%%% Local Variables:
%%% mode: latex
%%% TeX-master: t
%%% End:
