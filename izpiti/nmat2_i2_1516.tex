% Created 2016-07-01 pet 02:25
\documentclass[12pt]{article}
\usepackage[utf8]{inputenc}
\usepackage[T1]{fontenc}
\usepackage{mathpazo}
\usepackage{amsmath}
\usepackage{amssymb}
\usepackage{hyperref}
\usepackage[slovene]{babel}
%\author{Martin Vuk}
\date{\today}
\title{Izpit iz Numerične matematike}
\hypersetup{
 pdfauthor={Martin Vuk},
 pdftitle={Izpit iz Numerične matematika},
 pdfkeywords={},
 pdfsubject={},
 pdfcreator={Emacs 24.5.1 (Org mode 8.3.4)}, 
 pdflang={Slovene}}
\begin{document}

%\maketitle
%\tableofcontents
\begin{center}
\textbf{\huge{Izpit iz numerične matematike}}\\
\large{Ljubljana, 1. 7. 2016}
\end{center}
\bigskip

\begin{enumerate}
\item 
Dan je sistem \(2n\) linearnih enačb z \(2n\) neznankami: 
\[ -2^i x_{i} + x_{n+i} = b_i,\quad i=1,\ldots,n\]
in 
\[x_{i-n} - 2^{i-n} x_{i} = b_i,\quad i=n+1,\ldots,2n.\]
\begin{enumerate}
\item Poiščite rešitev sistema za \(n=2\) in \(b_i=i\).
\item Zapišite učinkovit in numerično stabilen algoritem za reševanje tega sistema.
Koliko operacij potrebuje vaš algoritem?
Poiščite rešitev sistema za \(n=20\) in \(b_i=i\).
\item Z inverzno potenčno metodo poiščite najmanjšo lastno vrednost matrike sistema
za \(n=20\).
\end{enumerate}

\item
Funkcijo 
\[f(x) = \arctan(x)\]
želimo na intervalu \([0,1]\) interpolirati s kubičnim polinomom. Naj bo \(p\)
polinom, ki interpolira podatke 
\[f(0), f(1/3), f(2/3)\text{ in }f(1),\] 
polinom \(q\) pa naj interpolira podatke
\[f(0), f'(0), f(1), f'(1).\]
\begin{enumerate}
\item Oba polinoma zapišite v Newtonovi obliki.
\item Numerično in analitično ocenite napako\footnote{Uporabite lahko dejstvo, da je \(f^(4)(x)=-\frac{24x(x^2-1)}{(x^2+1)^4}\).} za oba polinoma \(p\) in \(q\).
\item Uporabite polinom \(p\) ali \(q\) in poiščite rešitev enačbe
\[\arctan(x)=1-x.\]
Rešitev dobljeno z interpolacijskim polinomom primerjajte s pravo rešitvijo. 
\end{enumerate}
\end{enumerate}
\end{document}