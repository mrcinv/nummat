\documentclass[slovene]{article}
\usepackage[T1]{fontenc}
\usepackage[utf8x]{inputenc}
\usepackage{amsmath}
\DeclareMathOperator*\sn{sn}
\usepackage{listings}
\lstset{basicstyle={\footnotesize},
frame=lines,
language=Octave,
literate={{č}{{\v{c}}}1 {š}{{\v{s}}}1 {ž}{{\v{z}}}1}}
\usepackage{textcomp}
\usepackage{babel}
\begin{document}
\begin{center}
\section*{Izpit iz Numerične matematike}

Ljubljana, 23. 6. 2015
\end{center}
\pagestyle{empty}

\begin{enumerate}
\item Linearen sistem $n=2k$ enačb
  \begin{equation}
    \label{eq:sistem}
    2x_i+x_{n-i+1}=1.\quad i=1,\ldots,n
  \end{equation}
  \begin{enumerate}
  \item Napišite razširjeno matriko za $n=4$ in rešitev sistema poiščite z LU razcepom.
  \item Za $n=20$ rešite sistem (\ref{eq:sistem}) z Jacobijevo iteracijo
      \begin{equation*}
        \label{eq:Jacobi}
        x_i^{(k+1)} = \frac{1}{a_{ii}}\left(b_i-\sum_{j\not= i}a_{ij}x_j^{(k)}\right).
      \end{equation*}
Koliko korakov iteracije potrebujemo za 3 pravilne decimalke, če za začetni približek izberemo vektor desnih strani? Koliko operacij potrebujemo za vsako iteracijo? 
  \end{enumerate}

\item Naj bo $y(x)$ rešitev diferencialne enačbe
  \begin{equation*}
    \label{eq:NDE}
    y''-y=-x,
  \end{equation*}
ki zadošča začetnemu pogoju $y(0)=-1$ in $y'(0)=2$.
\begin{enumerate}
\item Izračunajte približek za $y(1)$ z Eulerjevo metodo s korakom $h=0.5$. Preverite, da je $y(x)=x-e^{-x}$ in določite lokalno napako na 1. koraku in globalno napako za $y(1)$.
\item Približek za $y(1)$ poiščite še z metodo RK4 
  \begin{align*}
    k_1&=h f(x_n,y_n)\\
    k_2&=h f(x_n+h/2,y_n+k_1/2)\\
    k_3&=h f(x_n+h/2,y_n+k_2/2)\\
    k_4&=h f(x_n+h,y_n+k_3)\\
    y_{n+1}&=y_n+\frac{k_1+2k_2+2k_3+k_4}{6}
  \end{align*}
s korakom $h=0.1$.
\item Z linearno interpolacijo na podatkih iz prejšnje točke poiščite približek za ničlo. Ocenite napako vašega približka za ničlo in vašo oceno primerjajte z dejansko napako.
\end{enumerate}
\end{enumerate}
\end{document}

%%% Local Variables:
%%% mode: latex
%%% TeX-master: t
%%% End:
