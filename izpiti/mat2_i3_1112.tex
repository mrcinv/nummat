\documentclass[a4paper,10pt]{article}
\usepackage[utf8x]{inputenc}
\usepackage[slovene]{babel}
\usepackage{amsmath}
\usepackage{amsfonts}
\usepackage{amssymb}
%opening
\title{Drugi izpit iz Matematike 2}
\date{28. 8. 2012}

\begin{document}

\maketitle

\section*{Popravni kolokvij}
\begin{enumerate}
\item Naj bo $Z:\mathbb{R}^{3}\to\mathbb{R}^{3}$ linearna prslikava, ki
predstavlja zraljenje preko ravnine
\[
x-y+z=0.
\]
Poišči lastne vrednosti in lastne vektorje preslikave $Z$. Če obstaja
ortonormirana baza sestavljena iz lastnih vektorjev preslikave $Z$,
jo poišči. Določi $3\times3$ matriko, ki pripada preslikavi $Z$.
\item Poišči bazo jedra in slike preslikave podane z matriko
\[
\begin{pmatrix}1 & 0 & 1 & 0\\
1 & 1 & 0 & 0\\
1 & -1 & 2 & 0
\end{pmatrix}.
\]

\item Poišči točko na elipsi
\[
x^{2}+xy+y^{2}=1,
\]
ki je najbližje premici skozi točki $A(-3,0)$ in $B(0,-3)$.
\item Rešujemo enačbo
\[
\sin x=ax+b,
\]
za $a>1$ in $b\in\mathbb{R}$.

\begin{enumerate}
\item Pokaži, da ima enačba eno samo rešitev (funkcija $\sin x-ax-b$ je
monotona) in določi končen interval v odvisnosti od $a$ in $b$,
na katerem se rešitev nahaja.
\item Za $a=2$ in $b=1$ reši enačbo z Newtonovo metodo in z metodo navadne
iteracije za ekvivalentno enačbo
\[
x=\frac{1}{a}(\sin x-b).
\]

\item Določi hitrost konvergence za obe metodi.
\end{enumerate}


\end{enumerate}
\newpage
\section*{Teoretični del}
\begin{enumerate}
%%%%%%
\item a) Napi"si definicijo linearne odvisnosti vektorjev $a,b,c$. Ali
pozna"s kako karakterizacijo linearne odvisnosti? Naj bo $a=(1,1)$,
$b=(1,0), c=(2,3)$. Poi"s"ci take $\alpha, \beta,\gamma$, $\alpha\neq
0$, za katere velja $\alpha a+\beta b+\gamma c=0$. Kaj na podlagi
dobljene enakosti lahko sklepamo o linearni odvisnosti (neodvisnosti)
vektorjev $a,b,c$?


b) Napi"si definicijo linearne ogrinja"ce in dolo"ci linearno
ogrinja"co vektorjev $a,b,c$.


\item

a) Naj bo $V={\mathbb R}^2$ in naj bosta $a,b$ in $c,d$ dve bazi prostora $V$.
Napi"si definicijo matrike prehoda $A$ od stare baze $a,b$ k novi bazi
$c,d$. Izra"cunaj matriko $A$, "ce so

$$
a=\begin{pmatrix}1\\ 1\end{pmatrix}, b=\begin{pmatrix}0\\
1\end{pmatrix}, c=\begin{pmatrix}0\\
1\end{pmatrix},d=\begin{pmatrix}1\\ 0\end{pmatrix},
$$


b) Napi"si definicijo koordinat vektorja v bazi. Za vektor
$d=\begin{pmatrix}5\\ 2\end{pmatrix}$ poi"s"ci njegove koordinate v
bazih $a,b$ in $c,d$. Kako uporabimo matriko $A$ in koordinate v bazi
$a,b$ da dobimo koordinate v bazi $c,d$?

\item Naj bo ${f(x)=\displaystyle \sum_{n=1}^\infty \frac{1}{n}(x-1)^n}$. Dolo\v ci
\begin{enumerate}
\item konvergen\v cni polmer in obmo\v cje konvergence poten\v cne vrste in pa definicijsko obmo\v cje funkcije $f$,
\item funkcijo $f'(x)$ in njeno definicijsko obmo\v cje,
\item funkcijo $f(x)$.
\end{enumerate}

\item Zapi\v si definicijo gradienta funkcije $f(x,y,z)$. Za funkcijo $f(x,y,z)=xz-2xy$ izra\v cunaj njen gradient. Poi\v s\v ci \v se smer, v kateri bo v to\v cki $(1,1,1)$ funkcijska vrednost najhitreje nara\v s\v cala.


%%%%%%%

\end{enumerate}
\end{document}
