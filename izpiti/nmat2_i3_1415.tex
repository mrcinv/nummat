\documentclass[slovene,12pt]{article}
\usepackage[T1]{fontenc}
\usepackage[utf8x]{inputenc}
\usepackage{mathpazo}
\usepackage{amsmath}
\DeclareMathOperator*\sn{sn}
\usepackage{listings}
\lstset{basicstyle={\footnotesize},
frame=lines,
language=Octave,
literate={{č}{{\v{c}}}1 {š}{{\v{s}}}1 {ž}{{\v{z}}}1}}
\usepackage{textcomp}
\usepackage{babel}
\begin{document}
\begin{center}
\section*{Izpit iz Numerične matematike}

Ljubljana, 9. 9. 2015
\end{center}
\vskip2cm
\pagestyle{empty}

\begin{enumerate}
\item Izračunajte ploščino lika, ki ga omejujeta grafa funkcij
  \begin{equation}
    \label{eq:lik}
    y=2^x\text{ in } y=\cos(x^2)
  \end{equation}
na intervalu $x\in[-3,-2]$. Ploščino izračunajte na 5 decimalk natančno. Ocenite relativno in absolutno napako.

\item Dan je tridiagonalen sistem enačb
  \begin{eqnarray}
    \label{eq:sistem}
    -x_{i-2}+(i+1)x_i-x_{i+2}=b_i;\quad i=1,\ldots,n
  \end{eqnarray}
  za spremenljivke $x_1,x_2,\ldots x_n$. Pri tem predpostavimo, da so manjkajoče vrednosti  enake nič $x_{-1}=x_{0}=x_{n+1}=x_{n+2}=0$.
  \begin{enumerate}
  \item Zapišite učinkovit algoritem za reševanje zgornjega sistema. Utemeljite, zakaj je algoritem numerično stabilen.
  \item Z vašim algoritmom rešite sistem za $n=4$ in $n=20$ z vrednostmi $b_i=1$.
  \item Z inverzno iteracijo poiščite najmanjšo lastno vrednost matrike sistema za $n=20$.
  \end{enumerate}
\end{enumerate}
\end{document}

%%% Local Variables:
%%% mode: latex
%%% TeX-master: t
%%% End:
