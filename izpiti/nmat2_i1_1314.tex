\documentclass[slovene]{article}
\usepackage[T1]{fontenc}
\usepackage[utf8x]{inputenc}
\usepackage{amsmath}
\usepackage{listings}
\lstset{basicstyle={\footnotesize},
frame=lines,
language=Octave,
literate={{č}{{\v{c}}}1 {š}{{\v{s}}}1 {ž}{{\v{z}}}1}}
\usepackage{textcomp}
\usepackage{babel}
\begin{document}

\title{Izpit iz Numerične matematike}


\date{11.6.2014}
\maketitle
\begin{enumerate}
\item Dan je začetni problem za NDE  
\begin{equation}
y''-\frac{2}{x}y'+y=0;\quad y_{1}(\pi/2)=1,\quad y_{1}'(\pi/2)=0.\label{eq:hermite}
\end{equation}
katerega rešitev je 
\begin{equation}
y(x)=\cos(x)+x\sin(x).\label{eq:resitev}
\end{equation} 
% $$y(x)=C_1\left(\cos(x)+x\sin(x)\right)+C_2\left(\sin(x)-x\cos(x)\right).$$
\begin{enumerate}
\item Poišči približek za $y(\pi)$ z metodo prediktor-korektor s korakom $h=\pi/4$.
Za prediktor uporabi Eulerjevo metodo $y_{n+1}=y_n +hf(x_n,y_n)$ za korektor pa 
implicitno trapezno pravilo
\begin{equation*}
y_{n+1}=y_n +\frac{1}{2}h(f(x_n,y_n)-f(x_{n+1},y_{n+1}).
\end{equation*}
Izračunaj globalno napako za $y(\pi)$.  

Ali se napaka bistveno spremeni, če uporabimo več korakov iteracije 
za izračun korektorja?
\item
Rešitev problema (\ref{eq:hermite}) ima na intervalu $[0,\pi]$ ničlo. 
Zgolj z izračunanimi približki iz točke (a) čimbolj natančno določi 
ničlo? Približek primerjaj z ničlo, ki jo dobiš za točno rešitvev 
(\ref{eq:resitev}).    
\end{enumerate}

\item S QR iteracijo poišči lastne vrednosti in lastne vektorje matrike
  \begin{equation}
    \label{eq:matrika}
    A= \begin{pmatrix}
      1 & 2 & 3\\
      2 & -1 & -2\\
      3 & -2 & 3
    \end{pmatrix}.
  \end{equation}
S potenčno in inverzno potenčno metodo izračunaj še lastni vektor za 
največjo in najmanjšo lastno vrednost.

Primerjaj število potrebnih korakov iteracije za isto 
natančnost med potenčno metodo in QR iteracijo. 
 \end{enumerate}

\end{document}
