\documentclass{article}

\usepackage[utf8]{inputenc}
\begin{document}


Začetni problem 
\[y''(x)-x\,y(x)=0,\quad y(0)=0.355028\,y'(0)=-0.258819,\] 
rešujemo kot sistem dveh diferencialnih enačb prvega reda, ki ga lahko zapišemo
kot 
\[\mathbf{Y}'=A(x)\,\mathbf{Y},\quad \mathbf{Y}(0)=\mathbf{Y}_0
  =[y(0),y'(0)]^T.\] 
Za računanje vrednosti \(y(x)\) uporabimo Magnusovo metodo reda 4, pri kateri
nov približek \(\mathbf{Y}_{k+1}\) dobimo takole:
\[\begin{array}{ccc}
    A_1&=&A\left(x_k+\left(\frac{1}{2}-\frac{\sqrt{3}}{6}\right)h\right)\\
    A_2&=&A\left(x_k+\left(\frac{1}{2}+\frac{\sqrt{3}}{6}\right)h\right)\\
    \sigma_{k+1}&=&\frac{h}{2}(A_1+A_2)-\frac{\sqrt{3}}{12}h^2[A_1,A_2]\\
    \mathbf{Y}_{k+1}&=&\exp(\sigma_{k+1})\mathbf{Y}_k.
\end{array}\] 
Izraz \([A,B]\) je komutator dveh matrik in ga izračunamo takole:
\([A,B]=AB-BA\). 
Eksponentno funkcijo na matriki (\(\exp(\sigma_{k+1})\)) pa v matlabu dobite z ukazom \texttt{expm}.
\end{document}